% Awesome Source CV LaTeX Template
%
% This template has been downloaded from:
% https://github.com/darwiin/awesome-neue-latex-cv
%
% Author:
% Christophe Roger
%
% Template license:
% CC BY-SA 4.0 (https://creativecommons.org/licenses/by-sa/4.0/)

%Section: Work Experience at the top
\sectionTitle{Experience}{\faSuitcase}
%\renewcommand{\labelitemi}{$\bullet$}
\begin{experiences}
  \experience
    {Present}   {Media Tech Architect}{\href{https://www.cablevisionfibertel.com.ar/}{\color{accentcolor}Grupo Telecom | Cablevisión}}{Buenos Aires - Argentina}
    {Dec 2018} {
                      \begin{itemize}
                      \item Migration of video headends to full cloud based services by keeping in mind new microservices architectures within containerized frameworks such as Docker/Kubernets
                      \item Architectures over the cloud (AWS) for traditional and new video services
                      \item Driving research and Probes of Concepts on leading video technologies in order to define new system architectures and optimizations. Specifically I focus on mid and long term evolution of video services
                      \item New video technologies and video issues:
                        \begin{itemize}
                        \item Quality of Experience and visual quality assesment techniques (VMAF, MOS, non-reference video metrics)
                        \item Encoding (AVC, HEVC, AV1)
                        \item Packaging (DASH, HLS, CMAF) 
                        \item ABR optimization: Video Players, CDN optimizations, ABR algorithms, bitrate eficiency, low latency architectures
                        \end{itemize}
                      \end{itemize}
                    }
                    {OTT, IPTV, CDN, QoS/E, Docker, Kubernetes, Linux, Python, C++, javascript}
  \emptySeparator
  \experience
    {Dec 2018} {Access Network  Engineer}{\href{https://www.cablevisionfibertel.com.ar/}{\color{accentcolor}Grupo Telecom | Cablevisión}}{Buenos Aires - Argentina}
    {Sep 2016}    {
                      \begin{itemize}
                      \item Analysis of the impact of new services over access networks technologies such as Docsis 3.0, Docsis 3.1, xDSL and GPON
                      \item Definitions of best practices for access  and home network monitoring by taking into account the QoS issues on services such as IPTV, VoIP and OTT
                      \end{itemize}
                    }
                    {Docsis,GPON,IPTV,OTT,QAM, QoS, SNPM, Python}
  \emptySeparator
  \experience
    {Dec 2018}     {Research Assistant}{\href{https://cnet.fi.uba.ar/en/}{\color{accentcolor}CoNexDat Lab} | Facultad de Ingeniería - UBA}{Buenos Aires - Argentina}
    {Jan 2014}    {
                      \begin{itemize}
                        \item Research over Internet Measurements and Internet Topology field.
                        \item Improvements and adaptation of open source tools for Internet Meassurements
                        \item Scripting for  measurements automation over large scale and distributed networks
                      \end{itemize}
                    }
                    {Networking, TCP/IP, QoS, CAIDA, scamper, tracebox, Python, R, C++, Linux, SQL}
  \emptySeparator
  \experience
  {Jul 2016}       {Presales Engineer}{XN}{Buenos Aires, Argentina}
  {Dec 2014}   {
                      \begin{itemize}
                      \item Product management for IoT  and smart cities solutions.
                      \item Design, evaluation and quotation of telecommunications infrastructure: wireless access networks, p2p RF links, enterprise solutions and video surveillance systems.
                      \item Preparation and presentation of technical offers in response to customers requirements.
                      \end{itemize}
                    }
                    {IoT,Networking, Wireless, CCTV, LoRa, ZigBee}
  \emptySeparator         
  \experience
  {Dec 2014}  {Technical Consultant}{Argentina Conectada Project}{Buenos Aires - Argentina}
  {Jun 2011}   {
                      \emph{Argentina Conectada} was a Government Project under the umbrella of the former \emph{Ministerio de Planificación} oriented to define the technical strategy to improve the coverage and the access to Internet.
                      \begin{itemize}
                        \item Technical advisoring for the deployment of a wide country data network
                        \item Technical feasibility analisys and high level desing over networking technologies such as:
                        \begin{itemize}
                        \item DWDM transport network (backbone)
                        \item IP/MPLS networks (backbone and aggregation)
                        \item FTTX´s terrestrial access technologies (access)                                            
                        \end{itemize}
                      \end{itemize}
                  }
                  {DWDM, IP/MPLS, FTTH, Fiber Optics}  

  \emptySeparator         
  \experience
  {Aug 2010}  {Support Engineer}{DESCA}{Quito - Ecuador}
  {Dec 2009}   {
                      \begin{itemize}
                        \item Networking Troubleshooting (Switching and Routing)
                        \item Configuring and Installation of Carrier Class Networking equipment
                        \item LAN network design and configuration
                        \item Firewall and network monitoring tools management
                      \end{itemize}
                }
                  {CCNA, Cisco, Routing, Switching, Networking, Firewall} 
  \emptySeparator         
  \experience
  {Aug 2010}  {Summer trainne}{Global Crossing}{Quito - Ecuador}
  {Dec 2009}   {
                Training and Intership in the Customer Engineering department of the former Global Crossing Company (Currently Level 3)
                }
                  {Routing, IP protocols, Networking} 
\end{experiences}

